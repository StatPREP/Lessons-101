\documentclass[nofonts,]{tufte-handout}

% ams
\usepackage{amssymb,amsmath}

\usepackage{ifxetex,ifluatex}
\usepackage{fixltx2e} % provides \textsubscript
\ifnum 0\ifxetex 1\fi\ifluatex 1\fi=0 % if pdftex
  \usepackage[T1]{fontenc}
  \usepackage[utf8]{inputenc}
\else % if luatex or xelatex
  \makeatletter
  \@ifpackageloaded{fontspec}{}{\usepackage{fontspec}}
  \makeatother
  \defaultfontfeatures{Ligatures=TeX,Scale=MatchLowercase}
  \makeatletter
  \@ifpackageloaded{soul}{
     \renewcommand\allcapsspacing[1]{{\addfontfeature{LetterSpace=15}#1}}
     \renewcommand\smallcapsspacing[1]{{\addfontfeature{LetterSpace=10}#1}}
   }{}
  \makeatother
\fi

% graphix
\usepackage{graphicx}
\setkeys{Gin}{width=\linewidth,totalheight=\textheight,keepaspectratio}

% booktabs
\usepackage{booktabs}

% url
\usepackage{url}

% hyperref
\usepackage{hyperref}

% units.
\usepackage{units}


\setcounter{secnumdepth}{-1}

% citations
\usepackage{natbib}
\bibliographystyle{plainnat}

%% tint override
\setcitestyle{round} 

% pandoc syntax highlighting

% longtable

% multiplecol
\usepackage{multicol}

% strikeout
\usepackage[normalem]{ulem}

% morefloats
\usepackage{morefloats}


% tightlist macro required by pandoc >= 1.14
\providecommand{\tightlist}{%
  \setlength{\itemsep}{0pt}\setlength{\parskip}{0pt}}

% title / author / date
\title{Variation and the 95\% summary interval}
\date{StatPREP Class Lesson}

%% -- tint overrides
%% fonts, using roboto (condensed) as default
\usepackage[sfdefault,condensed]{roboto}
%% also nice: \usepackage[default]{lato}

%% colored links, setting 'borrowed' from RJournal.sty with 'Thanks, Achim!'
\RequirePackage{color}
\definecolor{link}{rgb}{0.1,0.1,0.8} %% blue with some grey
\hypersetup{
  colorlinks,%
  citecolor=link,%
  filecolor=link,%
  linkcolor=link,%
  urlcolor=link
}

%% macros
\makeatletter

%% -- tint does not use italics or allcaps in title
\renewcommand{\maketitle}{%     
  \newpage
  \global\@topnum\z@% prevent floats from being placed at the top of the page
  \begingroup
    \setlength{\parindent}{0pt}%
    \setlength{\parskip}{4pt}%
    \let\@@title\@empty
    \let\@@author\@empty
    \let\@@date\@empty
    \ifthenelse{\boolean{@tufte@sfsidenotes}}{%
      %\gdef\@@title{\sffamily\LARGE\allcaps{\@title}\par}%
      %\gdef\@@author{\sffamily\Large\allcaps{\@author}\par}%
      %\gdef\@@date{\sffamily\Large\allcaps{\@date}\par}%
      \gdef\@@title{\begingroup\fontseries{b}\selectfont\LARGE{\@title}\par}%
      \gdef\@@author{\begingroup\fontseries{l}\selectfont\Large{\@author}\par}%
      \gdef\@@date{\begingroup\fontseries{l}\selectfont\Large{\@date}\par}%
    }{%
      %\gdef\@@title{\LARGE\itshape\@title\par}%
      %\gdef\@@author{\Large\itshape\@author\par}%
      %\gdef\@@date{\Large\itshape\@date\par}%
      \gdef\@@title{\begingroup\fontseries{b}\selectfont\LARGE\@title\par\endgroup}%
      \gdef\@@author{\begingroup\fontseries{l}\selectfont\Large\@author\par\endgroup}%
      \gdef\@@date{\begingroup\fontseries{l}\selectfont\Large\@date\par\endgroup}%
    }%
    \@@title
    \@@author
    \@@date
  \endgroup
  \thispagestyle{plain}% suppress the running head
  \tuftebreak% add some space before the text begins
  \@afterindentfalse\@afterheading% suppress indentation of the next paragraph
}

%% -- tint does not use italics or allcaps in section/subsection/paragraph
\titleformat{\section}%
  [hang]% shape
  %{\normalfont\Large\itshape}% format applied to label+text
  {\fontseries{b}\selectfont\Large}% format applied to label+text
  {\thesection}% label
  {1em}% horizontal separation between label and title body
  {}% before the title body
  []% after the title body

\titleformat{\subsection}%
  [hang]% shape
  %{\normalfont\large\itshape}% format applied to label+text
  {\fontseries{m}\selectfont\large}% format applied to label+text
  {\thesubsection}% label
  {1em}% horizontal separation between label and title body
  {}% before the title body
  []% after the title body

\titleformat{\paragraph}%
  [runin]% shape
  %{\normalfont\itshape}% format applied to label+text
  {\fontseries{l}\selectfont}% format applied to label+text
  {\theparagraph}% label
  {1em}% horizontal separation between label and title body
  {}% before the title body
  []% after the title body

%% -- tint does not use italics here either
% Formatting for main TOC (printed in front matter)
% {section} [left] {above} {before w/label} {before w/o label} {filler + page} [after]
\ifthenelse{\boolean{@tufte@toc}}{%
  \titlecontents{part}% FIXME
    [0em] % distance from left margin
    %{\vspace{1.5\baselineskip}\begin{fullwidth}\LARGE\rmfamily\itshape} % above (global formatting of entry)
    {\vspace{1.5\baselineskip}\begin{fullwidth}\fontseries{m}\selectfont\LARGE} % above (global formatting of entry)
    {\contentslabel{2em}} % before w/label (label = ``II'')
    {} % before w/o label
    {\rmfamily\upshape\qquad\thecontentspage} % filler + page (leaders and page num)
    [\end{fullwidth}] % after
  \titlecontents{chapter}%
    [0em] % distance from left margin
    %{\vspace{1.5\baselineskip}\begin{fullwidth}\LARGE\rmfamily\itshape} % above (global formatting of entry)
    {\vspace{1.5\baselineskip}\begin{fullwidth}\fontseries{m}\selectfont\LARGE} % above (global formatting of entry)
    {\hspace*{0em}\contentslabel{2em}} % before w/label (label = ``2'')
    {\hspace*{0em}} % before w/o label
    %{\rmfamily\upshape\qquad\thecontentspage} % filler + page (leaders and page num)
    {\upshape\qquad\thecontentspage} % filler + page (leaders and page num)
    [\end{fullwidth}] % after
  \titlecontents{section}% FIXME
    [0em] % distance from left margin
    %{\vspace{0\baselineskip}\begin{fullwidth}\Large\rmfamily\itshape} % above (global formatting of entry)
    {\vspace{0\baselineskip}\begin{fullwidth}\fontseries{m}\selectfont\Large} % above (global formatting of entry)
    {\hspace*{2em}\contentslabel{2em}} % before w/label (label = ``2.6'')
    {\hspace*{2em}} % before w/o label
    %{\rmfamily\upshape\qquad\thecontentspage} % filler + page (leaders and page num)
    {\upshape\qquad\thecontentspage} % filler + page (leaders and page num)
    [\end{fullwidth}] % after
  \titlecontents{subsection}% FIXME
    [0em] % distance from left margin
    %{\vspace{0\baselineskip}\begin{fullwidth}\large\rmfamily\itshape} % above (global formatting of entry)
    {\vspace{0\baselineskip}\begin{fullwidth}\fontseries{m}\selectfont\large} % above (global formatting of entry)
    {\hspace*{4em}\contentslabel{4em}} % before w/label (label = ``2.6.1'')
    {\hspace*{4em}} % before w/o label
    %{\rmfamily\upshape\qquad\thecontentspage} % filler + page (leaders and page num)
    {\upshape\qquad\thecontentspage} % filler + page (leaders and page num)
    [\end{fullwidth}] % after
  \titlecontents{paragraph}% FIXME
    [0em] % distance from left margin
    %{\vspace{0\baselineskip}\begin{fullwidth}\normalsize\rmfamily\itshape} % above (global formatting of entry)
    {\vspace{0\baselineskip}\begin{fullwidth}\fontseries{m}\selectfont\normalsize\rmfamily} % above (global formatting of entry)
    {\hspace*{6em}\contentslabel{2em}} % before w/label (label = ``2.6.0.0.1'')
    {\hspace*{6em}} % before w/o label
    %{\rmfamily\upshape\qquad\thecontentspage} % filler + page (leaders and page num)
    {\upshape\qquad\thecontentspage} % filler + page (leaders and page num)
    [\end{fullwidth}] % after
}{}

  
\makeatother



\begin{document}

\maketitle




latex

\href{/word-versions/summary-interval.docx}{Word version}

\hypertarget{orientation}{%
\section{Orientation}\label{orientation}}

Why is the word \emph{variable} used to describe a column of a data
frame? In mathematics, a variable is an \emph{unknown} quantity. In math
class, when they say ``solve for x,'' they mean figure out what the
value of unknown x is.

In statistics, a \emph{variable} is different. It's a known quantity --
known because you measured it, known because there it is, right in the
data frame. This known quantity is called a variable because it takes on
various different values in one row or another of the data frame.

An important technical task in statistics is to \emph{measure} how much
variation there is in a variable. The reason this is important is that
we are often interested in \emph{explaining} a response variable using
one or more explanatory variables. In deciding whether a potential
explanation is a good one or not, it's helpful to know how much of the
variation in the response can be attributed to the explanatory
variables.

There are several ways to measure variation of a quantitative variable.
The most commonly used in statistics is the \emph{standard deviation}.
Another measure is the \emph{95\% summary interval}, which is perhaps
easier to understand than the standard deviation.

The purpose of this lesson is to help you understand the 95\% summary
interval and why it is used by statisticians.

\hypertarget{activity}{%
\section{Activity}\label{activity}}

Open up the
\href{https://dtkaplan.shinyapps.io/LA_center_spread/}{Center and
spread} Little App (see footnote\footnote{\url{https://dtkaplan.shinyapps.io/LA_center_spread/}}).,
and select the \texttt{Births\_2014} data frame. Set the sample size to
n = 200.

\begin{enumerate}
\def\labelenumi{\arabic{enumi}.}
\item
  Set the response variable to be \texttt{age\_mother}, the age of the
  mother at the time of birth. You'll see that there is variability,
  that is, not every mother is exactly the same age.
\item
  Perhaps the most natural way to measure the amount of variation is the
  length of the interval between the smallest value and the largest
  value.

  \begin{itemize}
  \tightlist
  \item
    Use the measuring stick built in to the app to find the length of
    the interval from smallest to largest age.
  \end{itemize}
\end{enumerate}

\emph{How long is it?}
~~.~~.~~.~~.~~.~~.~~.~~.~~.~~.~~.~~.~~.~~.~~.~~.~~.~~.~~.~~.~~.~~.~~.~~.~~.
~

\emph{What are the units?}
~~.~~.~~.~~.~~.~~.~~.~~.~~.~~.~~.~~.~~.~~.~~.~~.~~.~~.~~.~~.~~.~~.~~.~~.~~.
~

\begin{enumerate}
\def\labelenumi{\arabic{enumi}.}
\setcounter{enumi}{2}
\item
  Leaving the measuring stick from (2) on the display, press the New
  Sample button. For the new sample, are the maximum and minimum values
  the same as in the original sample?

  \begin{itemize}
  \tightlist
  \item
    Press New Sample several times to get an idea of how much the
    maximum and minimum value vary from sample to sample.
  \item
    Do that again 10 times, but each time you do so use the measuring
    stick to find the length of the max-to-min interval for that sample.
    Write it down. In the end, you'll have 10 values for the length of
    the interval. You can get an idea for how much they vary.
  \end{itemize}

  \emph{List your 10 values here:} ~

  ~

  ~
\item
  Sometimes variables are such that the values of the maximum and
  minimum depend strongly on the sample. Sometimes not. One at a time,
  set each of these variables to be the response variable: mother's
  weight, baby's weight, APGAR score.

  \begin{itemize}
  \tightlist
  \item
    For each of the variables, press New Sample several times. The
    max-to-min interval will change from sample to sample. Use the
    measuring stick to find the length of each interval.
  \end{itemize}

  \emph{Write down your interval lengths here.} ~

  ~

  \emph{What's the range of the interval lengths you measured?} ~

  ~
\item
  You might have noticed that only two of the points in each sample
  determine the max-to-min interval. For this reason, there is a lot of
  variation in the interval length from sample to sample. To avoid this,
  statisticians have invented another kind of interval to describe
  variation called the ``95\% summary interval.'' The idea is to avoid
  making the interval depend on just two points and instead let several
  points contribute. The 95\% interval is designed to cover \emph{almost
  all} of the data points, where ``almost all'' means 95\%.

  \begin{itemize}
  \tightlist
  \item
    Turn on the ``summary interval'' display and make sure the
    ``interval level'' is set to 95\%. Keep the sample size at n = 200.
  \end{itemize}

  \emph{How many of the data points are \textbf{outside} the summary
  interval?} ~~.~~.~~.~~.~~.~~.~~.~~.~~.~~.~~.~~.~~.~~.~~. ~

  \emph{Explain why this number makes sense.} (Hint: Think how the 95\%
  interval is defined.) ~~.~~.~~.~~.~~.~~.~~.~~.~~.~~.~~.~~.~~.~~.~~. ~

  ~

  ~

  \begin{itemize}
  \tightlist
  \item
    Press New Sample many times to get a sense for whether the 95\%
    interval or the max-to-min interval varies more from sample to
    sample.
  \end{itemize}

  \emph{Which interval varies more?}
  ~~.~~.~~.~~.~~.~~.~~.~~.~~.~~.~~.~~.~~.~~.~~. ~

  ~

  ~
\item
  In statistics, you're almost always working with a sample but your
  real interest is not that particular sample, but the broader
  population from which the sample was taken. Statisticians like to work
  with summaries that are informative but which also behave ``nicely''
  with respect to sampling. In particular,

  \begin{itemize}
  \tightlist
  \item
    The summary shouldn't vary excessively from sample to sample.
  \item
    The summary shouldn't change systematically as the sample size gets
    larger. You've already looked at how the max-to-min interval and the
    95\% summary interval vary from one sample to another, holding the
    sample size constant. (We used n = 200.) Now you're going to see
    what happens as the sample size get's bigger and bigger.
  \end{itemize}

  \begin{enumerate}
  \def\labelenumii{\alph{enumii}.}
  \item
    Set the sample size to n = 200 and turn on the display of the 95\%
    summary interval. Quickly press New Sample several times to get an
    idea of a typical value for the maximum and minimum value of the
    variable in the sample. Then, use the measuring stick to mark out
    roughly the range from the typical maximum to the typical minimum.

    \emph{What's a typical range of the max-to-min interval?}
    ~~.~~.~~.~~.~~. ~
  \item
    Similarly, find the typical length of the 95\% summary interval over
    many samples. We only have one measuring stick in the app, so use
    your finger widths to record that typical length. (E.g. the interval
    is 3 fingers wide, or whatever.)

    \emph{What's a typical range of the 95\% interval?} ~~.~~.~~.~~.~~.
    ~
  \item
    Now increase the sample size to n = 1000. Press New Sample several
    times and keep track of how often the max-to-min interval is shorter
    than the one shown by the measuring stick for n = 200, and how often
    it is longer.

    What's a typical range of the max-to-min interval? ~~.~~.~~.~~.~~. ~
  \item
    Similarly compare the length of the 95\% summary interval when n =
    1000 to the length of the interval you recorded with your fingers in
    step (b).

    \emph{Circle the correct choice.} ~~.~~.~~.

    \emph{Compared to the n = 200 interval, the n = 1000 interval is
    about \ldots{} the same length \ldots{} twice as long \ldots{} half
    as long.}
  \item
    Do the same with a sample size of n = 5000.

    \emph{Which is more consistent across sample sizes, the 95\%
    interval or the max-to-min interval?} ~~.~~.~~.~~.~~.~~.~~.~~.~~.~~.
    ~
  \end{enumerate}

  ~
\end{enumerate}

\begin{center}\rule{0.5\linewidth}{\linethickness}\end{center}

Version 0.3, 2019-05-28, Daniel Kaplan,



\end{document}
